% Options for packages loaded elsewhere
\PassOptionsToPackage{unicode}{hyperref}
\PassOptionsToPackage{hyphens}{url}
%
\documentclass[
]{article}
\usepackage{lmodern}
\usepackage{amssymb,amsmath}
\usepackage{ifxetex,ifluatex}
\ifnum 0\ifxetex 1\fi\ifluatex 1\fi=0 % if pdftex
  \usepackage[T1]{fontenc}
  \usepackage[utf8]{inputenc}
  \usepackage{textcomp} % provide euro and other symbols
\else % if luatex or xetex
  \usepackage{unicode-math}
  \defaultfontfeatures{Scale=MatchLowercase}
  \defaultfontfeatures[\rmfamily]{Ligatures=TeX,Scale=1}
\fi
% Use upquote if available, for straight quotes in verbatim environments
\IfFileExists{upquote.sty}{\usepackage{upquote}}{}
\IfFileExists{microtype.sty}{% use microtype if available
  \usepackage[]{microtype}
  \UseMicrotypeSet[protrusion]{basicmath} % disable protrusion for tt fonts
}{}
\makeatletter
\@ifundefined{KOMAClassName}{% if non-KOMA class
  \IfFileExists{parskip.sty}{%
    \usepackage{parskip}
  }{% else
    \setlength{\parindent}{0pt}
    \setlength{\parskip}{6pt plus 2pt minus 1pt}}
}{% if KOMA class
  \KOMAoptions{parskip=half}}
\makeatother
\usepackage{xcolor}
\IfFileExists{xurl.sty}{\usepackage{xurl}}{} % add URL line breaks if available
\IfFileExists{bookmark.sty}{\usepackage{bookmark}}{\usepackage{hyperref}}
\hypersetup{
  pdftitle={JClub 4/26 Updates},
  pdfauthor={Ben},
  hidelinks,
  pdfcreator={LaTeX via pandoc}}
\urlstyle{same} % disable monospaced font for URLs
\usepackage[margin=1in]{geometry}
\usepackage{color}
\usepackage{fancyvrb}
\newcommand{\VerbBar}{|}
\newcommand{\VERB}{\Verb[commandchars=\\\{\}]}
\DefineVerbatimEnvironment{Highlighting}{Verbatim}{commandchars=\\\{\}}
% Add ',fontsize=\small' for more characters per line
\usepackage{framed}
\definecolor{shadecolor}{RGB}{248,248,248}
\newenvironment{Shaded}{\begin{snugshade}}{\end{snugshade}}
\newcommand{\AlertTok}[1]{\textcolor[rgb]{0.94,0.16,0.16}{#1}}
\newcommand{\AnnotationTok}[1]{\textcolor[rgb]{0.56,0.35,0.01}{\textbf{\textit{#1}}}}
\newcommand{\AttributeTok}[1]{\textcolor[rgb]{0.77,0.63,0.00}{#1}}
\newcommand{\BaseNTok}[1]{\textcolor[rgb]{0.00,0.00,0.81}{#1}}
\newcommand{\BuiltInTok}[1]{#1}
\newcommand{\CharTok}[1]{\textcolor[rgb]{0.31,0.60,0.02}{#1}}
\newcommand{\CommentTok}[1]{\textcolor[rgb]{0.56,0.35,0.01}{\textit{#1}}}
\newcommand{\CommentVarTok}[1]{\textcolor[rgb]{0.56,0.35,0.01}{\textbf{\textit{#1}}}}
\newcommand{\ConstantTok}[1]{\textcolor[rgb]{0.00,0.00,0.00}{#1}}
\newcommand{\ControlFlowTok}[1]{\textcolor[rgb]{0.13,0.29,0.53}{\textbf{#1}}}
\newcommand{\DataTypeTok}[1]{\textcolor[rgb]{0.13,0.29,0.53}{#1}}
\newcommand{\DecValTok}[1]{\textcolor[rgb]{0.00,0.00,0.81}{#1}}
\newcommand{\DocumentationTok}[1]{\textcolor[rgb]{0.56,0.35,0.01}{\textbf{\textit{#1}}}}
\newcommand{\ErrorTok}[1]{\textcolor[rgb]{0.64,0.00,0.00}{\textbf{#1}}}
\newcommand{\ExtensionTok}[1]{#1}
\newcommand{\FloatTok}[1]{\textcolor[rgb]{0.00,0.00,0.81}{#1}}
\newcommand{\FunctionTok}[1]{\textcolor[rgb]{0.00,0.00,0.00}{#1}}
\newcommand{\ImportTok}[1]{#1}
\newcommand{\InformationTok}[1]{\textcolor[rgb]{0.56,0.35,0.01}{\textbf{\textit{#1}}}}
\newcommand{\KeywordTok}[1]{\textcolor[rgb]{0.13,0.29,0.53}{\textbf{#1}}}
\newcommand{\NormalTok}[1]{#1}
\newcommand{\OperatorTok}[1]{\textcolor[rgb]{0.81,0.36,0.00}{\textbf{#1}}}
\newcommand{\OtherTok}[1]{\textcolor[rgb]{0.56,0.35,0.01}{#1}}
\newcommand{\PreprocessorTok}[1]{\textcolor[rgb]{0.56,0.35,0.01}{\textit{#1}}}
\newcommand{\RegionMarkerTok}[1]{#1}
\newcommand{\SpecialCharTok}[1]{\textcolor[rgb]{0.00,0.00,0.00}{#1}}
\newcommand{\SpecialStringTok}[1]{\textcolor[rgb]{0.31,0.60,0.02}{#1}}
\newcommand{\StringTok}[1]{\textcolor[rgb]{0.31,0.60,0.02}{#1}}
\newcommand{\VariableTok}[1]{\textcolor[rgb]{0.00,0.00,0.00}{#1}}
\newcommand{\VerbatimStringTok}[1]{\textcolor[rgb]{0.31,0.60,0.02}{#1}}
\newcommand{\WarningTok}[1]{\textcolor[rgb]{0.56,0.35,0.01}{\textbf{\textit{#1}}}}
\usepackage{graphicx,grffile}
\makeatletter
\def\maxwidth{\ifdim\Gin@nat@width>\linewidth\linewidth\else\Gin@nat@width\fi}
\def\maxheight{\ifdim\Gin@nat@height>\textheight\textheight\else\Gin@nat@height\fi}
\makeatother
% Scale images if necessary, so that they will not overflow the page
% margins by default, and it is still possible to overwrite the defaults
% using explicit options in \includegraphics[width, height, ...]{}
\setkeys{Gin}{width=\maxwidth,height=\maxheight,keepaspectratio}
% Set default figure placement to htbp
\makeatletter
\def\fps@figure{htbp}
\makeatother
\setlength{\emergencystretch}{3em} % prevent overfull lines
\providecommand{\tightlist}{%
  \setlength{\itemsep}{0pt}\setlength{\parskip}{0pt}}
\setcounter{secnumdepth}{-\maxdimen} % remove section numbering
\usepackage{booktabs}
\usepackage{longtable}
\usepackage{array}
\usepackage{multirow}
\usepackage{wrapfig}
\usepackage{float}
\usepackage{colortbl}
\usepackage{pdflscape}
\usepackage{tabu}
\usepackage{threeparttable}
\usepackage{threeparttablex}
\usepackage[normalem]{ulem}
\usepackage{makecell}
\usepackage{xcolor}

\title{JClub 4/26 Updates}
\author{Ben}
\date{4/26/2021}

\begin{document}
\maketitle

\hypertarget{jc-project-426}{%
\subsection{JC Project 4/26}\label{jc-project-426}}

Ben's list of things to do for Journal club prior to meeting on 4/29

\begin{enumerate}
\def\labelenumi{\arabic{enumi}.}
\tightlist
\item
  Generate new datasets for crop subsets
\item
  Generate full corrplot for relevant responses
\item
  Summarize Models for 1995-2015 on interactive effects of Nitrogen,
  Exp/Imp, and Pesticide use on area changes
\item
  Re-examine Z-level Factors non-transformed
\end{enumerate}

\hypertarget{generate-new-datasets-for-crop-subsets}{%
\subsubsection{1. Generate new datasets for crop
subsets}\label{generate-new-datasets-for-crop-subsets}}

\hypertarget{goal-have-datasets-available-where-we-can-examine-effects-on-only-a-subset-of-staple-crops-i.e.-those-primarily-gm-in-certain-countries}{%
\paragraph{Goal: Have datasets available where we can examine effects on
only a subset of staple crops, i.e.~those primarily GM in certain
countries}\label{goal-have-datasets-available-where-we-can-examine-effects-on-only-a-subset-of-staple-crops-i.e.-those-primarily-gm-in-certain-countries}}

From what I've seen, the main 4 GM crops grown globally are soybean,
cotton, maize, and canola, and since canola isn't one of those staple
crops I'm going to create a few datasets looking at the other 3 together
and individuallly.

Big disclaimer here is that ``cotton'' is actually ``Seed cotton'' aka
calories from cottonseed oil, so that may be unrepresentative when it
comes to actual cotton production or yield calculations.

Not shown here, but have datasets now available that have each of the
three individually and combined for analysis.

Quick example, lets look at the 1979-1999 and 1995-2015 Ewers
relationships if it was ONLY SOYBEANS! (made another new little
function, only other change is ``area.tot'' is now ``area.harv'')

\hypertarget{just-soy}{%
\paragraph{Just Soy}\label{just-soy}}

\begin{Shaded}
\begin{Highlighting}[]
\NormalTok{soy1 <-}\StringTok{ }\KeywordTok{ewers_plot_avg_commod}\NormalTok{(}\DataTypeTok{data =}\NormalTok{ soy_full }\OperatorTok\StringTok{ }\KeywordTok{filter}\NormalTok{(Area }\OperatorTok{!=}\StringTok{ "Brunei Darussalam"}\NormalTok{, Area }\OperatorTok{!=}\StringTok{ "Maldives"}\NormalTok{), }
                   \DataTypeTok{X =} \StringTok{"kcal.ha.avg"}\NormalTok{,}
                   \DataTypeTok{Y =} \StringTok{"area.harv"}\NormalTok{,}
                   \DataTypeTok{start =} \DecValTok{1979}\NormalTok{,}
                   \DataTypeTok{end =} \DecValTok{1999}\NormalTok{) }
\NormalTok{soy2 <-}\StringTok{ }\KeywordTok{ewers_plot_avg_commod}\NormalTok{(}\DataTypeTok{data =}\NormalTok{ soy_full }\OperatorTok\StringTok{ }\KeywordTok{filter}\NormalTok{(Area }\OperatorTok{!=}\StringTok{ "Brunei Darussalam"}\NormalTok{, Area }\OperatorTok{!=}\StringTok{ "Maldives"}\NormalTok{), }
                   \DataTypeTok{X =} \StringTok{"kcal.ha.avg"}\NormalTok{,}
                   \DataTypeTok{Y =} \StringTok{"area.harv"}\NormalTok{,}
                   \DataTypeTok{start =} \DecValTok{1995}\NormalTok{,}
                   \DataTypeTok{end =} \DecValTok{2015}\NormalTok{)}
\KeywordTok{ggarrange}\NormalTok{(soy1,soy2, }\DataTypeTok{common.legend =}\NormalTok{ T)}
\end{Highlighting}
\end{Shaded}

\includegraphics{JC_update_4_29_files/figure-latex/unnamed-chunk-1-1.pdf}

Interesting but unsurprising that individual commodities dont follow
global trends!

\hypertarget{generate-full-corrplot-for-relevant-responses}{%
\subsection{2. Generate full corrplot for relevant
responses}\label{generate-full-corrplot-for-relevant-responses}}

Looking first at untransformed values here.

\hypertarget{section}{%
\paragraph{1995}\label{section}}

\begin{Shaded}
\begin{Highlighting}[]
\KeywordTok{corrplot}\NormalTok{(}\KeywordTok{cor}\NormalTok{(}\KeywordTok{drop_na}\NormalTok{(fao_updated_staple }\OperatorTok\StringTok{ }\KeywordTok{filter}\NormalTok{(Year }\OperatorTok{==}\StringTok{ }\DecValTok{1995}\NormalTok{) }\OperatorTok\StringTok{ }
\StringTok{               }\NormalTok{dplyr}\OperatorTok{::}\KeywordTok{select}\NormalTok{(}\DecValTok{3}\NormalTok{,tonnes_nitrogen,CO2_eq_emissions,tonnes_pest_total,kcal.ha.avg,mean.staple.exports,mean.staple.imports))))}
\end{Highlighting}
\end{Shaded}

\includegraphics{JC_update_4_29_files/figure-latex/1995-1.pdf}

\hypertarget{section-1}{%
\paragraph{2015}\label{section-1}}

\begin{Shaded}
\begin{Highlighting}[]
\KeywordTok{corrplot}\NormalTok{(}\KeywordTok{cor}\NormalTok{(}\KeywordTok{drop_na}\NormalTok{(fao_updated_staple }\OperatorTok\StringTok{ }\KeywordTok{filter}\NormalTok{(Year }\OperatorTok{==}\StringTok{ }\DecValTok{2015}\NormalTok{) }\OperatorTok\StringTok{ }
\StringTok{               }\NormalTok{dplyr}\OperatorTok{::}\KeywordTok{select}\NormalTok{(}\DecValTok{3}\NormalTok{,tonnes_nitrogen,CO2_eq_emissions,tonnes_pest_total,kcal.ha.avg,mean.staple.exports,mean.staple.imports))))}
\end{Highlighting}
\end{Shaded}

\includegraphics{JC_update_4_29_files/figure-latex/unnamed-chunk-2-1.pdf}

\hypertarget{per-capita}{%
\paragraph{1995 per capita}\label{per-capita}}

Below two plots are raw values / Population

\begin{Shaded}
\begin{Highlighting}[]
\KeywordTok{corrplot}\NormalTok{(}\KeywordTok{cor}\NormalTok{(}\KeywordTok{drop_na}\NormalTok{(fao_updated_staple }\OperatorTok\StringTok{ }\KeywordTok{filter}\NormalTok{(Year }\OperatorTok{==}\StringTok{ }\DecValTok{1995}\NormalTok{) }\OperatorTok\StringTok{ }
\StringTok{               }\NormalTok{dplyr}\OperatorTok{::}\KeywordTok{select}\NormalTok{(}\DecValTok{3}\NormalTok{,tonnes_nitrogen,CO2_eq_emissions,tonnes_pest_total,kcal.ha.avg,mean.staple.exports,mean.staple.imports) }\OperatorTok\StringTok{ }
\StringTok{              }\NormalTok{dplyr}\OperatorTok{::}\KeywordTok{mutate}\NormalTok{(}\DataTypeTok{tonnes_nitrogen.pc =}\NormalTok{ tonnes_nitrogen}\OperatorTok{/}\NormalTok{Population,}
                            \DataTypeTok{CO2_eq_emissions.pc =}\NormalTok{ CO2_eq_emissions}\OperatorTok{/}\NormalTok{Population,}
                            \DataTypeTok{tonnes_pest_total.pc =}\NormalTok{ tonnes_pest_total}\OperatorTok{/}\NormalTok{Population,}
                            \DataTypeTok{kcal.ha.avg.pc =}\NormalTok{ kcal.ha.avg}\OperatorTok{/}\NormalTok{Population,}
                            \DataTypeTok{mean.staple.exports.pc =}\NormalTok{ mean.staple.exports}\OperatorTok{/}\NormalTok{Population,}
                            \DataTypeTok{mean.staple.imports.pc =}\NormalTok{ mean.staple.imports}\OperatorTok{/}\NormalTok{Population) }\OperatorTok\StringTok{ }
\StringTok{                }\NormalTok{dplyr}\OperatorTok{::}\KeywordTok{select}\NormalTok{(}\OperatorTok{-}\DecValTok{2}\OperatorTok{:-}\DecValTok{7}\NormalTok{))))}
\end{Highlighting}
\end{Shaded}

\includegraphics{JC_update_4_29_files/figure-latex/unnamed-chunk-3-1.pdf}

\hypertarget{per-capita-1}{%
\paragraph{2015 per capita}\label{per-capita-1}}

\begin{Shaded}
\begin{Highlighting}[]
\KeywordTok{corrplot}\NormalTok{(}\KeywordTok{cor}\NormalTok{(}\KeywordTok{drop_na}\NormalTok{(fao_updated_staple }\OperatorTok\StringTok{ }\KeywordTok{filter}\NormalTok{(Year }\OperatorTok{==}\StringTok{ }\DecValTok{2015}\NormalTok{) }\OperatorTok\StringTok{ }
\StringTok{               }\NormalTok{dplyr}\OperatorTok{::}\KeywordTok{select}\NormalTok{(}\DecValTok{3}\NormalTok{,tonnes_nitrogen,CO2_eq_emissions,tonnes_pest_total,kcal.ha.avg,mean.staple.exports,mean.staple.imports) }\OperatorTok\StringTok{ }
\StringTok{              }\NormalTok{dplyr}\OperatorTok{::}\KeywordTok{mutate}\NormalTok{(}\DataTypeTok{tonnes_nitrogen.pc =}\NormalTok{ tonnes_nitrogen}\OperatorTok{/}\NormalTok{Population,}
                            \DataTypeTok{CO2_eq_emissions.pc =}\NormalTok{ CO2_eq_emissions}\OperatorTok{/}\NormalTok{Population,}
                            \DataTypeTok{tonnes_pest_total.pc =}\NormalTok{ tonnes_pest_total}\OperatorTok{/}\NormalTok{Population,}
                            \DataTypeTok{kcal.ha.avg.pc =}\NormalTok{ kcal.ha.avg}\OperatorTok{/}\NormalTok{Population,}
                            \DataTypeTok{mean.staple.exports.pc =}\NormalTok{ mean.staple.exports}\OperatorTok{/}\NormalTok{Population,}
                            \DataTypeTok{mean.staple.imports.pc =}\NormalTok{ mean.staple.imports}\OperatorTok{/}\NormalTok{Population) }\OperatorTok\StringTok{ }
\StringTok{                }\NormalTok{dplyr}\OperatorTok{::}\KeywordTok{select}\NormalTok{(}\OperatorTok{-}\DecValTok{2}\OperatorTok{:-}\DecValTok{7}\NormalTok{))))}
\end{Highlighting}
\end{Shaded}

\includegraphics{JC_update_4_29_files/figure-latex/unnamed-chunk-4-1.pdf}

\hypertarget{summarize-all-existing-models-for-1995-2015-on-interactive-effects-of-nitrogen-expimp-and-pesticide-use-on-area-changes}{%
\subsection{3. Summarize All Existing Models for 1995-2015 on
interactive effects of Nitrogen, Exp/Imp, and Pesticide use on area
changes}\label{summarize-all-existing-models-for-1995-2015-on-interactive-effects-of-nitrogen-expimp-and-pesticide-use-on-area-changes}}

\hypertarget{figure-1-per-capita-change-in-area-used-to-grow-and-yield-of-23-staple-crops}{%
\subsubsection{Figure 1: Per capita change in area used to grow and
yield of 23 staple
crops}\label{figure-1-per-capita-change-in-area-used-to-grow-and-yield-of-23-staple-crops}}

First Figure is the basic comparison of Ewers 2009 style analysis with
updated 1995-2015 data. Reminder, our method now includes averaging of
two years before and after stated time (i.e.~``1995'' is the average of
1993-1997)
\includegraphics{JC_update_4_29_files/figure-latex/unnamed-chunk-5-1.pdf}

These models represent effects of ``kcal.ha.avg'' (yield metric) on area
used to grow staple crops \#\#\#\# Model Summaries

\begin{Shaded}
\begin{Highlighting}[]
    \CommentTok{# Old Ewers Model (Area Change ~ Yield Change 1979-1999)}
\KeywordTok{Anova}\NormalTok{(}\KeywordTok{lm}\NormalTok{(log.y.dif }\OperatorTok{~}\StringTok{ }\NormalTok{log.x.dif, }\DataTypeTok{data =}\NormalTok{ ewers_}\DecValTok{79}\NormalTok{_}\DecValTok{99}\NormalTok{_data))}
\end{Highlighting}
\end{Shaded}

\begin{verbatim}
## Anova Table (Type II tests)
## 
## Response: log.y.dif
##            Sum Sq  Df F value Pr(>F)
## log.x.dif  0.1617   1  0.8523 0.3574
## Residuals 28.4500 150
\end{verbatim}

\begin{Shaded}
\begin{Highlighting}[]
    \CommentTok{# New Ewers Model (Area Change ~ Yield Change 1995-2015)}
\KeywordTok{Anova}\NormalTok{(}\KeywordTok{lm}\NormalTok{(log.y.dif }\OperatorTok{~}\StringTok{ }\NormalTok{log.x.dif, }\DataTypeTok{data =}\NormalTok{ ewers_}\DecValTok{95}\NormalTok{_}\DecValTok{15}\NormalTok{_data))}
\end{Highlighting}
\end{Shaded}

\begin{verbatim}
## Anova Table (Type II tests)
## 
## Response: log.y.dif
##           Sum Sq  Df F value    Pr(>F)    
## log.x.dif  2.887   1  11.843 0.0007248 ***
## Residuals 42.414 174                      
## ---
## Signif. codes:  0 '***' 0.001 '**' 0.01 '*' 0.05 '.' 0.1 ' ' 1
\end{verbatim}

We find a slightly negative but non-significant relationship from the
1979-1999 period between yield and total area, but a significant
positive relationship from 1995-2015.

As a reminder, Ewers found a slight negative relationship in their
analysis, so we will need to address this later given our difference
there. I think our year averaging technique is superior, but may need to
incorporate what they did for the Soviet Union (we have like 40 more
countries than them too).

\hypertarget{hdi-faceting-same-values}{%
\paragraph{HDI Faceting same Values}\label{hdi-faceting-same-values}}

\includegraphics{JC_update_4_29_files/figure-latex/Ewers style faceted-1.pdf}

This shows that positive relation is driven primarily by High and Very
high HDI contries, and could help justify our investigation of countries
by HDI and other explaining factors.

\hypertarget{nitrogen}{%
\subsubsection{Nitrogen}\label{nitrogen}}

First Figure is a recreation of the Ewers style plots, with change in
nitrogen use on the Z. Model summaries below that look at the
interaction between changes in yield and changes in nitrogen use on area
used to grow staple crops.

\includegraphics{JC_update_4_29_files/figure-latex/Nitrogen Data gen and plots-1.pdf}

\begin{Shaded}
\begin{Highlighting}[]
\CommentTok{# 1979-1999 Area ~ Yield * Nitrogen}
\KeywordTok{tidy}\NormalTok{(}\KeywordTok{lm}\NormalTok{(log.y.dif }\OperatorTok{~}\StringTok{ }\NormalTok{log.x.dif}\OperatorTok{*}\NormalTok{log.z.dif, }\DataTypeTok{data =}\NormalTok{ data79_}\DecValTok{99}\NormalTok{_nitro))}
\end{Highlighting}
\end{Shaded}

\begin{verbatim}
## # A tibble: 4 x 5
##   term                estimate std.error statistic       p.value
##   <chr>                  <dbl>     <dbl>     <dbl>         <dbl>
## 1 (Intercept)          -0.300     0.0460    -6.52  0.00000000186
## 2 log.x.dif            -0.120     0.114     -1.05  0.295        
## 3 log.z.dif             0.0998    0.0578     1.73  0.0870       
## 4 log.x.dif:log.z.dif   0.0440    0.144      0.306 0.760
\end{verbatim}

\begin{Shaded}
\begin{Highlighting}[]
\CommentTok{# 1995-2015 Area ~ Yield * Nitrogen}
\KeywordTok{tidy}\NormalTok{(}\KeywordTok{lm}\NormalTok{(log.y.dif }\OperatorTok{~}\StringTok{ }\NormalTok{log.x.dif}\OperatorTok{*}\NormalTok{log.z.dif, }\DataTypeTok{data =}\NormalTok{ data95_}\DecValTok{15}\NormalTok{_nitro))}
\end{Highlighting}
\end{Shaded}

\begin{verbatim}
## # A tibble: 4 x 5
##   term                estimate std.error statistic  p.value
##   <chr>                  <dbl>     <dbl>     <dbl>    <dbl>
## 1 (Intercept)          -0.262     0.0355    -7.38  1.63e-11
## 2 log.x.dif             0.222     0.120      1.85  6.63e- 2
## 3 log.z.dif             0.140     0.0408     3.43  7.98e- 4
## 4 log.x.dif:log.z.dif   0.0242    0.106      0.229 8.20e- 1
\end{verbatim}

\hypertarget{area-changes-nitrogen-use-1995-2015}{%
\subsubsection{Area Changes \textasciitilde{} Nitrogen Use
1995-2015}\label{area-changes-nitrogen-use-1995-2015}}

\includegraphics{JC_update_4_29_files/figure-latex/unnamed-chunk-8-1.pdf}

\begin{verbatim}
## Anova Table (Type II tests)
## 
## Response: log.y.dif
##            Sum Sq  Df F value    Pr(>F)    
## log.z.dif  1.8703   1  12.469 0.0005691 ***
## Residuals 19.9489 133                      
## ---
## Signif. codes:  0 '***' 0.001 '**' 0.01 '*' 0.05 '.' 0.1 ' ' 1
\end{verbatim}

\hypertarget{same-again-split-by-hdi-this-time}{%
\paragraph{Same again Split by HDI this
time}\label{same-again-split-by-hdi-this-time}}

\includegraphics{JC_update_4_29_files/figure-latex/unnamed-chunk-9-1.pdf}

\begin{verbatim}
## Anova Table (Type II tests)
## 
## Response: log.y.dif
##                Sum Sq  Df F value    Pr(>F)    
## log.z.dif      1.9467   1 13.1401 0.0004168 ***
## HDI            0.2227   3  0.5010 0.6822503    
## log.z.dif:HDI  0.9113   3  2.0505 0.1101305    
## Residuals     18.8149 127                      
## ---
## Signif. codes:  0 '***' 0.001 '**' 0.01 '*' 0.05 '.' 0.1 ' ' 1
\end{verbatim}

\hypertarget{imports-and-exports}{%
\subsubsection{Imports and Exports}\label{imports-and-exports}}

\hypertarget{exports}{%
\paragraph{Exports}\label{exports}}

\includegraphics{JC_update_4_29_files/figure-latex/Exp Data and First Ewers-1.pdf}

\^{}\^{}\^{} Lots of missing values above

Analysis

\begin{Shaded}
\begin{Highlighting}[]
\CommentTok{# 1979-1999 Area ~ Yield * Exports}
\KeywordTok{tidy}\NormalTok{(}\KeywordTok{lm}\NormalTok{(log.y.dif }\OperatorTok{~}\StringTok{ }\NormalTok{log.x.dif}\OperatorTok{*}\NormalTok{log.z.dif, }\DataTypeTok{data =}\NormalTok{ data79_}\DecValTok{99}\NormalTok{_exp))}
\end{Highlighting}
\end{Shaded}

\begin{verbatim}
## # A tibble: 4 x 5
##   term                estimate std.error statistic p.value
##   <chr>                  <dbl>     <dbl>     <dbl>   <dbl>
## 1 (Intercept)         -0.213      0.0782   -2.73   0.00836
## 2 log.x.dif           -0.122      0.208    -0.586  0.560  
## 3 log.z.dif           -0.0100     0.0460   -0.218  0.828  
## 4 log.x.dif:log.z.dif  0.00164    0.0971    0.0169 0.987
\end{verbatim}

\begin{Shaded}
\begin{Highlighting}[]
\CommentTok{# 1995-2015 Area ~ Yield * Exports}
\KeywordTok{tidy}\NormalTok{(}\KeywordTok{lm}\NormalTok{(log.y.dif }\OperatorTok{~}\StringTok{ }\NormalTok{log.x.dif}\OperatorTok{*}\NormalTok{log.z.dif, }\DataTypeTok{data =}\NormalTok{ data95_}\DecValTok{15}\NormalTok{_exp))}
\end{Highlighting}
\end{Shaded}

\begin{verbatim}
## # A tibble: 4 x 5
##   term                estimate std.error statistic      p.value
##   <chr>                  <dbl>     <dbl>     <dbl>        <dbl>
## 1 (Intercept)         -0.373      0.0636   -5.86   0.0000000446
## 2 log.x.dif            0.308      0.175     1.76   0.0816      
## 3 log.z.dif            0.113      0.0346    3.27   0.00143     
## 4 log.x.dif:log.z.dif -0.00361    0.0883   -0.0409 0.967
\end{verbatim}

\hypertarget{imports}{%
\paragraph{Imports}\label{imports}}

\includegraphics{JC_update_4_29_files/figure-latex/unnamed-chunk-10-1.pdf}

\begin{Shaded}
\begin{Highlighting}[]
\CommentTok{# 1979-1999 Area ~ Yield * Imports}
\KeywordTok{tidy}\NormalTok{(}\KeywordTok{lm}\NormalTok{(log.y.dif }\OperatorTok{~}\StringTok{ }\NormalTok{log.x.dif}\OperatorTok{*}\NormalTok{log.z.dif, }\DataTypeTok{data =}\NormalTok{ data79_}\DecValTok{99}\NormalTok{_imp))}
\end{Highlighting}
\end{Shaded}

\begin{verbatim}
## # A tibble: 4 x 5
##   term                estimate std.error statistic      p.value
##   <chr>                  <dbl>     <dbl>     <dbl>        <dbl>
## 1 (Intercept)          -0.285     0.0474    -6.00  0.0000000160
## 2 log.x.dif            -0.0451    0.112     -0.403 0.687       
## 3 log.z.dif            -0.0399    0.0631    -0.632 0.529       
## 4 log.x.dif:log.z.dif  -0.143     0.151     -0.948 0.345
\end{verbatim}

\begin{Shaded}
\begin{Highlighting}[]
\CommentTok{# 1995-2015 Area ~ Yield * Imports}
\KeywordTok{tidy}\NormalTok{(}\KeywordTok{lm}\NormalTok{(log.y.dif }\OperatorTok{~}\StringTok{ }\NormalTok{log.x.dif}\OperatorTok{*}\NormalTok{log.z.dif, }\DataTypeTok{data =}\NormalTok{ data95_}\DecValTok{15}\NormalTok{_imp))}
\end{Highlighting}
\end{Shaded}

\begin{verbatim}
## # A tibble: 4 x 5
##   term                estimate std.error statistic     p.value
##   <chr>                  <dbl>     <dbl>     <dbl>       <dbl>
## 1 (Intercept)          -0.306     0.0591     -5.18 0.000000639
## 2 log.x.dif             0.557     0.136       4.09 0.0000665  
## 3 log.z.dif             0.0956    0.0486      1.96 0.0511     
## 4 log.x.dif:log.z.dif  -0.265     0.104      -2.55 0.0116
\end{verbatim}

\hypertarget{area-changes-exportsimports-1995-2015}{%
\paragraph{Area Changes \textasciitilde{} Exports/Imports
1995-2015}\label{area-changes-exportsimports-1995-2015}}

\includegraphics{JC_update_4_29_files/figure-latex/solo Exp/Imp Plots-1.pdf}

\begin{Shaded}
\begin{Highlighting}[]
\CommentTok{# 1995-2015 Area Change ~ Change in Exports}
\KeywordTok{tidy}\NormalTok{(}\KeywordTok{lm}\NormalTok{(log.y.dif }\OperatorTok{~}\StringTok{ }\NormalTok{log.z.dif, }\DataTypeTok{data =}\NormalTok{ data95_}\DecValTok{15}\NormalTok{_exp))}
\end{Highlighting}
\end{Shaded}

\begin{verbatim}
## # A tibble: 2 x 5
##   term        estimate std.error statistic      p.value
##   <chr>          <dbl>     <dbl>     <dbl>        <dbl>
## 1 (Intercept)   -0.396    0.0648     -6.11 0.0000000130
## 2 log.z.dif      0.134    0.0342      3.91 0.000153
\end{verbatim}

\begin{Shaded}
\begin{Highlighting}[]
\CommentTok{# 1995-2015 Area Change ~ Change in Imports}
\KeywordTok{tidy}\NormalTok{(}\KeywordTok{lm}\NormalTok{(log.y.dif }\OperatorTok{~}\StringTok{ }\NormalTok{log.z.dif, }\DataTypeTok{data =}\NormalTok{ data95_}\DecValTok{15}\NormalTok{_imp))}
\end{Highlighting}
\end{Shaded}

\begin{verbatim}
## # A tibble: 2 x 5
##   term        estimate std.error statistic  p.value
##   <chr>          <dbl>     <dbl>     <dbl>    <dbl>
## 1 (Intercept)   -0.382    0.0583     -6.56 6.49e-10
## 2 log.z.dif      0.149    0.0484      3.07 2.46e- 3
\end{verbatim}

\hypertarget{exportimport-split-by-hdi}{%
\paragraph{Export/Import Split by HDI}\label{exportimport-split-by-hdi}}

\includegraphics{JC_update_4_29_files/figure-latex/unnamed-chunk-11-1.pdf}
Analyses

\begin{Shaded}
\begin{Highlighting}[]
\CommentTok{# 1995-2015 Area Change ~ Change in Exports * HDI}
\KeywordTok{Anova}\NormalTok{(}\KeywordTok{lm}\NormalTok{(log.y.dif }\OperatorTok{~}\StringTok{ }\NormalTok{log.z.dif}\OperatorTok{*}\NormalTok{HDI, }\DataTypeTok{data =}\NormalTok{ data95_}\DecValTok{15}\NormalTok{_exp))}
\end{Highlighting}
\end{Shaded}

\begin{verbatim}
## Anova Table (Type II tests)
## 
## Response: log.y.dif
##                Sum Sq  Df F value    Pr(>F)    
## log.z.dif      2.4888   1 14.5071 0.0002284 ***
## HDI            0.0487   3  0.0947 0.9628510    
## log.z.dif:HDI  0.6312   3  1.2263 0.3035552    
## Residuals     19.2145 112                      
## ---
## Signif. codes:  0 '***' 0.001 '**' 0.01 '*' 0.05 '.' 0.1 ' ' 1
\end{verbatim}

\begin{Shaded}
\begin{Highlighting}[]
\CommentTok{# 1995-2015 Area Change ~ Change in Imports * HDI}
\KeywordTok{Anova}\NormalTok{(}\KeywordTok{lm}\NormalTok{(log.y.dif }\OperatorTok{~}\StringTok{ }\NormalTok{log.z.dif}\OperatorTok{*}\NormalTok{HDI, }\DataTypeTok{data =}\NormalTok{ data95_}\DecValTok{15}\NormalTok{_imp))}
\end{Highlighting}
\end{Shaded}

\begin{verbatim}
## Anova Table (Type II tests)
## 
## Response: log.y.dif
##               Sum Sq  Df F value   Pr(>F)   
## log.z.dif      2.077   1  8.4031 0.004265 **
## HDI            0.184   3  0.2477 0.862889   
## log.z.dif:HDI  1.808   3  2.4385 0.066465 . 
## Residuals     40.049 162                    
## ---
## Signif. codes:  0 '***' 0.001 '**' 0.01 '*' 0.05 '.' 0.1 ' ' 1
\end{verbatim}

\hypertarget{examine-z-factors-non-transformed}{%
\subsection{4. Examine Z factors
non-transformed}\label{examine-z-factors-non-transformed}}

This is now looking at some plots and figures without the Z level
responses (i.e.~nitrogen) scaled to per-capita. These still represent
log(late/early) values to reflect change over time, they values are just
not adjusted to population for each region. This may be a bit useless in
the long run, but might help us identify some of the biggest changes in
practices.

\hypertarget{section-2}{%
\paragraph{}\label{section-2}}

\includegraphics{JC_update_4_29_files/figure-latex/unnamed-chunk-13-1.pdf}
\includegraphics{JC_update_4_29_files/figure-latex/unnamed-chunk-13-2.pdf}
\includegraphics{JC_update_4_29_files/figure-latex/unnamed-chunk-13-3.pdf}

\begin{Shaded}
\begin{Highlighting}[]
\CommentTok{# 1995-2015 Area Change ~ Change in Nitrogen Applied (tonnes unscaled) * HDI}
\KeywordTok{Anova}\NormalTok{(}\KeywordTok{lm}\NormalTok{(log.y.dif }\OperatorTok{~}\StringTok{ }\NormalTok{log.z.dif}\OperatorTok{*}\NormalTok{HDI, }\DataTypeTok{data =}\NormalTok{ unscaled_nitro))}
\end{Highlighting}
\end{Shaded}

\begin{verbatim}
## Anova Table (Type II tests)
## 
## Response: log.y.dif
##                Sum Sq  Df F value   Pr(>F)   
## log.z.dif      2.3308   1 10.3990 0.001597 **
## HDI            0.0829   3  0.1232 0.946239   
## log.z.dif:HDI  1.8663   3  2.7756 0.043985 * 
## Residuals     28.9131 129                    
## ---
## Signif. codes:  0 '***' 0.001 '**' 0.01 '*' 0.05 '.' 0.1 ' ' 1
\end{verbatim}

\begin{Shaded}
\begin{Highlighting}[]
\CommentTok{# 1995-2015 Area Change ~ Change in Exports (unscaled) * HDI}
\KeywordTok{Anova}\NormalTok{(}\KeywordTok{lm}\NormalTok{(log.y.dif }\OperatorTok{~}\StringTok{ }\NormalTok{log.z.dif}\OperatorTok{*}\NormalTok{HDI, }\DataTypeTok{data =}\NormalTok{ unscaled_exports))}
\end{Highlighting}
\end{Shaded}

\begin{verbatim}
## Anova Table (Type II tests)
## 
## Response: log.y.dif
##                Sum Sq  Df F value    Pr(>F)    
## log.z.dif      2.7009   1 12.2645 0.0006683 ***
## HDI            0.3393   3  0.5136 0.6737628    
## log.z.dif:HDI  0.7015   3  1.0618 0.3684481    
## Residuals     24.2240 110                      
## ---
## Signif. codes:  0 '***' 0.001 '**' 0.01 '*' 0.05 '.' 0.1 ' ' 1
\end{verbatim}

\begin{Shaded}
\begin{Highlighting}[]
\CommentTok{# 1995-2015 Area Change ~ Change in Imports (unscaled) * HDI}
\KeywordTok{Anova}\NormalTok{(}\KeywordTok{lm}\NormalTok{(log.y.dif }\OperatorTok{~}\StringTok{ }\NormalTok{log.z.dif}\OperatorTok{*}\NormalTok{HDI, }\DataTypeTok{data =}\NormalTok{ unscaled_imports))}
\end{Highlighting}
\end{Shaded}

\begin{verbatim}
## Anova Table (Type II tests)
## 
## Response: log.y.dif
##               Sum Sq  Df F value Pr(>F)
## log.z.dif      0.391   1  1.4608 0.2286
## HDI            0.179   3  0.2230 0.8803
## log.z.dif:HDI  0.538   3  0.6690 0.5723
## Residuals     42.857 160
\end{verbatim}

At first glance there doesn't really seem to be that much of a
difference with or without scaling (given the ratio is log transformed)
so we may want to look into another way of investigating this. Instead
of the ration (late/early) we could simply look at one time points
completely raw or log-transformed values for correlations with change in
area (i.e.~only the countries with the most tonnes of nitrogen added had
changes to area).

\end{document}
